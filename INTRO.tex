\chapter[\mbox{O q faz Artaud pra acabar com o juizo de deus}\\por José Celso Martinez Corrêa]{O q faz Artaud\break pra acabar com\break o juizo de deus}

Em 4 de novembro de 1996, por ocasião dos 100 anos do nascimento
de Antonin Artaud, o Teat(r)o Oficina Uzyna Uzona comemorou esta data
com uma Pré Encenação, um Improviso no Teatro do Museu de
Arte de São Paulo (\textbf{MASP}) da Peça Radiofônica \emph{Pra dar
um Fim no Juízo de Deus}. Convidados pela Companhía
\textbf{TaanTeatro} criada por Maura Baioque, Talentosíssima
Dançarina, Antropófoga da Arte do Butô do Japão no Brasil e seu Duplo:
Wolfgang Pannek. Gratidão Éterna á essa Companhía, q
Transtornou o ``Teat(r)o Oficina, nos presenteando com a Produção de
\emph{Pra dar um Fim no Juízo de Deus} de Antonin Artaud.

\section*{O TEATRO ARTAUDIANO, OCULTO NO MASP}

``O Arquiteto'' Lina Bardi, (era assim q Éla se definia
na sua Profissão: no Masculino) Criou na sua Grande Obra no
\textbf{MASP} um dos mais Belos Espaços Teatrais do Mundo. Inspirada em
Antonin Artaud, Criou um Anel em \textbf{U} de Cimento ligado ás Paredes
do Espaço em forma de Ferradura q se encontra com o Palco, formando no
seu Todo o Chão da Atuação das Atrizes y Atores, Cantores, Dançarinos,
Cineastas.

Lina Inspirou-se Totalmente na Concepção Arquitetônica de Antonin Artaud
pra Cena Contemporânea, descrita em detalhes em ``O Teatro e seu
Duplo''. Um Palco Circundando o Público no Centro Sentado em Banquinhos.
O ``Arquitéto'' Lina Bardi desenhou ``Cadeiras Giratórias'', Para o
Público poder acompanhar as Ações das Atrizes Atores das Peças. Lina
criou as Paredes também como Telas de Cinema Projetando Filmes, q
aconteceriam neste envoltório de 360º.

Mas a Direção do \textsc{masp} transformou este \textbf{Extraordinário Teatro}
num \textbf{Auditório} com Poltronas Fixas em direção á um
\textbf{Palquinho.}

Lina virou uma \textbf{Fúria}, \textbf{veio uma Vermelhidão de
Fogo em seu Rosto: ``Colocaram lá, no Teatro de Artaud, um Palquinho e
uma Platéia de Colégio de Freiras. Nunca mais vou por os Pés neste
Lugar''.}

Pois nosso 1º Encontro com o Teatro de Artaud se deu
Neste Teatro Oculto com ``Poltronas Elefantes mirando o Palco
Italiano''. Ignoramos essa Elefantíase-Poltronal y Atuamos no Chão Mais
Alto, do Palco de 360º, cercando Totalmente o Público Sentadão!

Artaud Escreveu essa Peça no Pós Guerra, depois de estar Internado
Nove Anos em Hospícios, em q recebeu 51 Eletrochoques. Mas escreveu
muito para seus Médicos. Retornou á Paris, liberto no Pós Guerra Mundial
dos Manicômios. O Diretor da Radio Difusão Nacional Francesa entre Todes
Grandes Artistas Franceses convida
Artaud á Participar do Progama ``A Voz Dos Poetas'' q iria ao Ar dia 2
de Fevereiro de 1948. Quería neste texto trazer Toda sua Sabedoría,
chamada ``Loucura'' q Tinha Construído Nestes Anos e sobretudo no seu
Contato com os Índios Taraumaras no México, Macumbando no Peyote, Peidando e Gargalhando!

Conhecemos a Peça, Encenando pela 1ª Vez na Configuração
Artaudiana do Teatro Atual do \textbf{MASP}. Os Espectadores tinham q se
virar pros 4 Cantos do Espaço, pra acompanhar o Espetáculo pra conseguir
ver além das Cadeiras-Tronos Gigantes sobre um Grosso Tapete no Chão pra
Esconder o Cimento.

Obvio q apesar das Poltronas, nos apresentamos no Imenso Palco
Circular Revelando na Encenação assassinando o Teatrinho Italianinho.

Inaugurando sob Aplausos do Publico Presente este
Inacreditavelmente Belo Espaço, Escondido num Auditório ``Fake''. Neste
Prefácio, vocês poderão Imaginar lendo as Cenas como apreendemos sua
realização nas Condições Rheais de Revelação num dos mais Sofisticados
Teatros não só de SamPã mas do Mundo. Nunca tínhamos apresentado a Peça,
foi descoberta por nós nesta Ocasião pela Presença Cênica do Espaço do
Teatro da Crueldade q Lina Bardi fez pra Antonin Artaud a Partir do
Texto q está no ``Teatro e Seu Duplo''. Violentamente Inspirados neste
Novíssimo Espaço. Cada Cena em torno do Público foi a inspiração pra
Parirmos como se fosse na Rua, o Imortal Poema Teatral de Artaud.

A Passagem pro Teatro Oficina não foi difícil depois deste
aprendizado neste Teatro do Masp q Inauguramos pra SamPã mas q continua
até Hoje como se não tivéssemos Revelado sua Beleza Concreta. Ainda
Agóra, não passa de um Auditório.

\begin{itemize}
\item[---] \emph{Cena do lugar d Encontro na Mesa de Autóspsia onde era depositada
Ritualmente por uma Enfermeira Sonâmbula a Latinha com o Esperma Vivo,
doado pelo Menino da 1ª Comunhão depois de Gozar atrás de um Biombo
depois a Latinha com a Merda Cagada pelo \textbf{Artaud-Marat}, o do
Filme q fez com Abel Gance \textbf{Napoleão}.}

\item[---] \emph{o \textbf{Sangue} oferecido dos braços de um ou uma Espectadora
  na Mesa de Autópsia é Derramado numa Ferradura Alçada na Ponta de um
  Enorme Bastão pelos Tarauramaras, depois de terem Plantado o Delírio
  Ritual do \textbf{Peyote} q arrancaram da Terra com seus Dentes.}

\item[---] \emph{Artaud participou com os Índios Taraumaras de suas Rodas de
  Macumbas, nas Montanhas onde moram no México, Tomados pelo Peyote,
  Gargalhando y Peidando Ritualmete sem Parar.}
\end{itemize}

\emph{A Merda, o Cocô vai Pra uma Latinha Recipiente Também. O
Monge interpretado por Marcelo Drummond Consagra Com Seu
``Côro de Merda'' o Cocô na Sua Cena abrindo Ritualmente a
Latinha e com o \textbf{``Coro de Merda''.}}

\begin{verse}
\textbf{MONGE}\\
\emph{Deus é Merda?}\\
\emph{Si não For}\\
\emph{Não É.}
\end{verse}


\bigskip

\noindent\emph{Numa das Cenas desse 3º Ato Surge \textbf{Joana dÁrc}
Contracenando com o Monge, Lembra a Cena Magnifica em q pode se ver na
Net com a Grande Atriz Renée Falconetti contracenando com o Belo Jovem
Artaud no Papel de Monge sob direção de Theodore Dryee. O Monge tira da
Parede e passa o Crucifixo para Joana d'Árc, a Grande Atriz Joana Medeiros.}

\emph{O Contraregra Corta-lhe os Cabelos. Ela Vai para a Fogueira de
Refletores Apontados pra seus Pés. Onde Nasce o Fogo. Cada Cena tem suas
Glossolalías, Mas o desta Cena pode tornar-se Exemplar. A Emoção de
Joana dÁrc e do Monge quando Ela queima na Fogueira é tão Forte quanto
Inexprimível. Pro Publico viver a Cena, Tem q se ir além da Palavras.
ArTaud cria com uma Enxurrada de não palavras, de imprecações, mágicas
pra explodir, a Emoção em Energía Sonora e Visual: As Glossolalías.
Assim por exemplo:}

\pagebreak
\mbox{}\vspace*{\fill}
\begin{verse}
\emph{ratara ratara ratara}\\
\emph{atara tatara rana}\\
\emph{\ldots{} etc.}
\end{verse}
\mbox{}\vspace*{\fill}
\pagebreak

\emph{Nesta Cena a Castidade, a Cristandade do Corpo de Joana dÁrc
queimando vai revelando a Mulher na Santa, Ícone em Carne demasiadamente
Humana. O Fogo da Cena vai pro Corpo E Erotiza o Corpo Tesudo de
Joana no Fogo q Levanta a Cruz e a destrói em 2 Pedaços q Jamais se
Cruzarão.}

\emph{A 4ª Cena é Inspirada na Peça \textbf{BEATRICE CENCI} Artaud faz o
Papel do Patriarca Cenci q Faz uma Festa em seu Castelo e depois de
Todos Presentes Convidados Tranca os Portões de seu Palácio e começa a
Praticar Atos de Crueldade com seus Convidados. Chega prender sua
própria Filha Beatrice Cenci numa Roda de Tortura. Inspirado nesta Cena
Criou-se a Cena de duas Mulheres aprisionadas num Hospício Vestidas com
Enormes Camisas de Força sendo Perseguida por Enfermeires Cruéis q
Tentam e Conseguem Amarra-las na Camisa de Força}:

\begin{verse}
\textbf{AS~CENCI~AMARRADAS~PELA~CAMISA~DE~FORÇA}\\
\emph{É q me Espremiam}\\
\emph{até o Meu Corpo}\\
\emph{e até o Corpo}\\ \EP[2]
\emph{e foi aí q eu PEIDEI}\\
\emph{PEIDEI DE DESRAZÃO}\\
\emph{E FIZ TUDO EXPLODIR}
\end{verse}

\begin{verse}
\textbf{A ÚLTIMA CENA É COM ARTAUD VELHO}

\emph{Personagem q Viví em Todas Temporadas da Peça}\\
\emph{Em Cima de uma Cama Metálica Nú}\\
\emph{Na Cena em q Artaud descobre o Corpo sem Orgãos.}\\
\emph{pra Re-fazer a Anatomia.}
\end{verse}

\emph{Esses Princípios inspiram Inteiramente os Filósofos Deleuze e
Guatarí, q escrévem o} Anti Édipo \emph{q tem como sub-texto Explicito:}
Pra dar um Fim no Juizo de Deus.

\emph{Este Prefácio não pode deixar de Revelar q as Obras de Artaud com
esta Peça, tirada de sua Gravação Proibida do Programa ``A Voz dos
Poetas'', Mesmo Proibída em 1948 se tornou Entre Todas as Vozes Permitídas,
a Mais Presente nos Corpos Humanos de hoje, na Pós Pandemía, na Luta
pelos Territórios dos Indígenas y na Luta pela Vida no Planeta Terra
Agonizante, Artaud por ser o Poeta da Vida como Coisa Concreta, do
Corpo, das Coisas Humanas, dos Bichos, das Pédras q Vão Sendoldots{}}

\emph{É um Poeta Voz da TragiKomédiOrgya q vivemos nestes Tempos de Fim
de Mundo parindo Mundos Novos, no Corpo dos Indígenas, dos Negros, dxs
Trans, em q a Multidão Insurrecional á sua maneira é a Fonte da
Renovação na sua saída dos Buracos, á Luz do Sol. E vem aí}
HELIOGÁBALO \emph{a Obra Prima de Artaud em língua Brazileira.}

\bigskip

\hfill\emph{José Celso Martinez Corrêa}


\hfill\emph{Dia 2 de Novembro de 2021 DIA DOS}

\hfill\emph{MORTOS ETHERNOS EM NOSSAS MEMÓRIAS}

\hfill\emph{ONDE DEPOIS DE MORTOS VIVEREMOS}

\hfill\emph{ETERNOS TODOS NA ALMA, NO CORAÇÃO}

\hfill\emph{Y NA CUCA DOS Q VIEREM.}
