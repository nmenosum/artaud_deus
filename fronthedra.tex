\AtBeginDocument{%
\begingroup\pagestyle{empty}\raggedleft\parindent0pt

\thispagestyle{empty}
\begin{textblock*}{2.625in}(0pt,0pt)%
\vspace*{-3cm}
\hspace*{-3.1cm}\includegraphics[width=164.5mm]{./imgs/newfront1.pdf}
\end{textblock*}

\vfill
\endgroup
\clearpage

%% Créditos ------------------------------------------------------
\begingroup\raggedright
\thispagestyle{empty}
{\Fakt\large{\textbf{PRA DAR UM FIM NO JUÍZO DE DEUS}}}

título original: \textit{Pour en finir avec le jugement de dieu}

Antonin Artaud

\bigskip

{\scriptsize
© José Celso Martinez Corrêa, 2022

© n-1 edições, 2022

\textsc{isbn} 978-65-86941-84-5

\bigskip

Embora adote a maioria dos usos editoriais do âmbito
brasileiro, a\\ n-1 edições não segue necessariamente as
convenções das instituições\\ normativas, pois considera
a edição um trabalho de criação que deve\\ interagir com
a pluralidade de linguagens e a especificidade de cada\\
obra publicada.



\bigskip

\linha{título original}{\titulooriginal}
\linha{título em português}{\tituloportugues}
\linhalayout{coordenação editorial}{Peter Pál Pelbart e Ricardo Muniz Fernandes}
\linhalayout{direção de arte}{Ricardo Muniz Fernandes}
\linha{tradução}{\copyrighttraducao}
\linha{organização©}{\copyrightorganizacao}
\linha{prefácio©}{\copyrightintroducao}
\linha{ilustração©}{\copyrightilustracao}
\linha{edição consultada}{\edicaoconsultada}
\linha{primeira edição}{\primeiraedicao}
\linha{agradecimentos}{\agradecimentos}
\linha{indicação}{\indicacao}
\linha{direção de arte}{\direcaodearte}
\linha{coedição}{\coedicao}
\linha{assistência editorial}{\assistencia}
\linha{preparação}{\preparacao}
\linha{revisão}{\revisao}
\linhalayout{edição em \LaTeX}{Paulo Henrique Pompermaier}
\linha{iconografia}{\iconografia}
\linhalayout{capa e projeto gráfico}{Isabel Teixeira / Ateliê Fora de Esquadro}
\linha{imagem da capa}{\imagemcapa}%\smallskip



\bigskip
\bigskip
\bigskip

A reprodução parcial deste livro sem fins lucrativos,
para uso\\ privado ou coletivo, em qualquer meio
impresso ou eletrônico, está\\ autorizada, desde que
citada a fonte. Se for necessária a reprodução\\ na
íntegra, solicita-se entrar em contato com os editores.

\vfill

Contém atos de desdiagramação

\vfill

1ª edição | Março, 2022\\
\textbf{n-1edicoes.org}\\
\bigskip
}
\endgroup

%% Front ---------------------------------------------------------
\pagebreak
\pagestyle{empty}


\begin{textblock*}{2.625in}(0pt,0pt)%
\vspace*{-3cm}
\hspace*{-3.7cm}\includegraphics[width=164.5mm]{./imgs/newfront2.pdf}
\end{textblock*}

\par\clearpage%\endgroup

% Sumário -------------------------------------------------------
\pagebreak
{\begingroup\mbox{}\pagestyle{empty}
\thispagestyle{empty} 
\movetooddpage
\addtocontents{toc}{\protect\thispagestyle{empty}}
\mbox{}\vspace*{\fill}\tableofcontents*\mbox{}\vspace*{\fill}\clearpage\endgroup}

} % fim do AtBeginDocument

% Finais -------------------------------------------------------
\AtEndDocument{%

\pagebreak

\thispagestyle{empty}

\vspace*{\fill}
{\scriptsize
\begin{center}
\textbf{Dados Internacionais de Catalogação na Publicação (CIP) de acordo com ISBD}\\

\textbf{\hrule}
\end{center}
\noindent{}A785p \quad Artaud, Antonin\\

  

\hspace{1cm}Para dar um fim no juízo de Deus / Antonin Artaud ;

\quad\quad traduzido por José Celso Martinez Correa. - São Paulo :

\quad\quad n-1 edições, 2022.

\hspace{1cm}\pageref{lastpage} p. ; 14cm x 21cm.\\

\hspace{1cm}Inclui índice.

\hspace{1cm}ISBN: 978-65-86941-84-5\\

\hspace{1cm}1. Teatro. 2. Artes do corpo. I. Correa, José Celso Martinez. II. Título.\\

\noindent{}\mbox{  }\hspace{205.5pt}\versal{CDD} 792\vspace*{-.1cm}

\noindent{}2021-428 \hspace{175pt} \versal{CDU} 792\\

\textbf{\hrule}
\begin{center}

\textbf{Elaborado por Odilio Hilario Moreira Junior - CRB-8/9949}\\[6pt]
\textbf{Índice para catálogo sistemático:}\\\vspace*{-.3cm}
\end{center}
\hspace{3.05cm}1. Teatro 792

\noindent\hspace{3.05cm}2. Teatro 792

}

\vspace*{\fill}



\pagebreak
\begingroup
\thispagestyle{empty}
%\pagecolor{black}\afterpage{\nopagecolor}
%\color{white}
\vspace*{\fill}
\raggedright
\begin{quotation}
{\linespread{1.2}\Fakt\noindent O livro como imagem do mundo é de toda maneira uma ideia insípida. Na verdade não basta dizer Viva o múltiplo, grito de resto difícil de emitir. Nenhuma habilidade tipográfica, lexical ou mesmo sintática será suficiente para fazê-lo ouvir. É preciso fazer o múltiplo, não acrescentando sempre uma dimensão superior, mas, ao contrário, da maneira mais simples, com força de sobriedade, no nível das dimensões de que se dispõe, sempre n-1 (é somente assim que o uno faz parte do múltiplo, estando sempre subtraído dele). Subtrair o único da multiplicidade a ser constituída; escrever a n-1.\\[5pt]

\noindent{}Gilles Deleuze e Félix Guattari}
\end{quotation}
\vspace*{\fill}
\endgroup

\pagebreak
\begingroup
\thispagestyle{empty}
\pagecolor{black}\afterpage{\nopagecolor}
\color{white}
\vspace*{\fill}
\begin{center}
\textbf{\Large n-1edicoes.org} \label{lastpage}
\end{center}
\vspace*{\fill}

\endgroup

}
